\documentclass[12pt]{article}
\usepackage{amsmath, amssymb}
\usepackage{geometry}  
\geometry{a4paper, margin=2cm}

\begin{document}

\section{Derivative}

\subsection{}
Given:
\[
f(x) = \frac{1}{\ln(1+x)} - \frac{1}{x}, \quad x \in [0,1], \quad \lambda = \max_{x \in [0,1]} f(x), \quad \mu = \min_{x \in [0,1]} f(x)
\]

As $x \to 0^+$:
\begin{align*}
\ln(1+x) &= x - \frac{x^2}{2} + O(x^3) \\
\frac{1}{\ln(1+x)} &= \frac{1}{x} \cdot \frac{1}{1 - \frac{x}{2} + O(x^2)} = \frac{1}{x} \left(1 + \frac{x}{2} + O(x^2)\right) = \frac{1}{x} + \frac{1}{2} + O(x) \\
f(x) &= \left(\frac{1}{x} + \frac{1}{2}\right) - \frac{1}{x} + O(x) = \frac{1}{2} + O(x) \to \frac{1}{2}
\end{align*}

At $x = 1$:
\[
f(1) = \frac{1}{\ln 2} - 1
\]

The derivative is:
\[
f'(x) = -\frac{1}{(1+x)\ln^2(1+x)} + \frac{1}{x^2}
\]

Setting $f'(x) = 0$ gives:
\[
\frac{1}{x^2} = \frac{1}{(1+x)\ln^2(1+x)} \Rightarrow (1+x)\ln^2(1+x) = x^2
\]

Numerical analysis shows $f'(x) < 0$ on $[0,1]$, so $f(x)$ is strictly decreasing. Therefore:
\[
\lambda = \lim_{x \to 0^+} f(x) = \frac{1}{2}, \quad \mu = f(1) = \frac{1}{\ln 2} - 1
\]

\subsection{}
Given $f(x)$ has continuous second-order derivatives on $[a,b]$. Prove:
\[
\lim_{n \to \infty} n^{2} \left[ \int_{a}^{b} f(x)  dx - \frac{b-a}{n} \sum_{k=1}^{n} f \left( a + \frac{2k-1}{2n}(b-a) \right) \right] = \frac{(b-a)^{2}}{24} \left[ f'(b) - f'(a) \right]
\]

\textbf{Proof:} Consider the error for one subinterval $[x_{k-1}, x_k]$ where $x_k = a + \frac{k}{n}(b-a)$ and midpoint $c_k = a + \frac{2k-1}{2n}(b-a)$. By Taylor expansion around $c_k$:
\begin{align*}
f(x) &= f(c_k) + f'(c_k)(x-c_k) + \frac{f''(\xi_k)}{2}(x-c_k)^2 \\
\int_{x_{k-1}}^{x_k} f(x)dx &= f(c_k)h + \frac{f''(\eta_k)}{24}h^3
\end{align*}
where $h = \frac{b-a}{n}$.

Summing over all subintervals:
\[
\int_a^b f(x)dx - \frac{b-a}{n} \sum_{k=1}^n f(c_k) = \sum_{k=1}^n \frac{f''(\eta_k)}{24}h^3 = \frac{(b-a)^3}{24n^2} \cdot \frac{1}{n} \sum_{k=1}^n f''(\eta_k)
\]

As $n \to \infty$, $\frac{1}{n} \sum_{k=1}^n f''(\eta_k) \to \frac{1}{b-a} \int_a^b f''(x)dx = \frac{f'(b) - f'(a)}{b-a}$. Therefore:
\[
\lim_{n \to \infty} n^2 \left[ \int_a^b f(x)dx - \frac{b-a}{n} \sum_{k=1}^n f(c_k) \right] = \frac{(b-a)^2}{24} [f'(b) - f'(a)]
\]

\newpage

\section{Limits}

\subsection{}
Compute:
\[
I = \lim_{x \to 0} \frac{\ln(\cos x)}{x^2}
\]

Using Taylor expansion:
\begin{align*}
\cos x &= 1 - \frac{x^2}{2} + O(x^4) \\
\ln(\cos x) &= \ln\left(1 - \frac{x^2}{2} + O(x^4)\right) = -\frac{x^2}{2} + O(x^4) \\
I &= \lim_{x \to 0} \frac{-\frac{x^2}{2} + O(x^4)}{x^2} = -\frac{1}{2}
\end{align*}

\subsection{}
Compute:
\[
I = \lim_{x \to 0} \frac{\ln(\cos x + x \sin 2x)}{e^{x^{2}} - \sqrt[3]{1 - x^{2}}}
\]

Expand numerator:
\begin{align*}
\cos x &= 1 - \frac{x^2}{2} + O(x^4) \\
x \sin 2x &= 2x^2 - \frac{4}{3}x^4 + O(x^6) \\
\cos x + x \sin 2x &= 1 + \frac{3}{2}x^2 + O(x^4) \\
\ln(\cos x + x \sin 2x) &= \frac{3}{2}x^2 + O(x^4)
\end{align*}

Expand denominator:
\begin{align*}
e^{x^2} &= 1 + x^2 + O(x^4) \\
\sqrt[3]{1 - x^2} &= 1 - \frac{1}{3}x^2 + O(x^4) \\
e^{x^2} - \sqrt[3]{1 - x^2} &= \frac{4}{3}x^2 + O(x^4)
\end{align*}

Therefore:
\[
I = \lim_{x \to 0} \frac{\frac{3}{2}x^2 + O(x^4)}{\frac{4}{3}x^2 + O(x^4)} = \frac{3/2}{4/3} = \frac{9}{8}
\]

\newpage

\section{Integrals}

\subsection{}
\[
I = \int \sec x  dx
\]
Multiply numerator and denominator by $\sec x + \tan x$:
\[
I = \int \frac{\sec x (\sec x + \tan x)}{\sec x + \tan x}  dx = \int \frac{\sec^2 x + \sec x \tan x}{\sec x + \tan x}  dx
\]
Let $u = \sec x + \tan x$, then $du = (\sec x \tan x + \sec^2 x)dx$:
\[
I = \int \frac{du}{u} = \ln |u| + C = \ln |\sec x + \tan x| + C
\]

\subsection{}
\[
I = \int \frac{dx}{x^2 + a^2}
\]
Let $x = a \tan t$, $dx = a \sec^2 t  dt$:
\[
I = \int \frac{a \sec^2 t}{a^2 (\tan^2 t + 1)}  dt = \frac{1}{a} \int dt = \frac{1}{a} t + C = \frac{1}{a} \arctan\left(\frac{x}{a}\right) + C
\]

\subsection{}
\[
I = \int \frac{dx}{\sqrt{x^2 + a^2}}
\]
Let $x = a \tan t$, $dx = a \sec^2 t  dt$, $\sqrt{x^2 + a^2} = a \sec t$:
\[
I = \int \frac{a \sec^2 t}{a \sec t}  dt = \int \sec t  dt = \ln |\sec t + \tan t| + C
\]
Substitute back: $\sec t = \frac{\sqrt{x^2 + a^2}}{a}$, $\tan t = \frac{x}{a}$:
\[
I = \ln \left| \frac{x}{a} + \frac{\sqrt{x^2 + a^2}}{a} \right| + C = \ln \left| x + \sqrt{x^2 + a^2} \right| + C'
\]

\newpage

\section{Multiple Integrals}

\subsection{}
\[
D = \{(x,y) \mid x^{2} + y^{2} \leq \pi\}, \quad I = \iint_{D} \left( \sin(x^{2}) \cos(y^{2}) + x \sqrt{x^{2} + y^{2}} \right)  dx  dy
\]

The second term $x \sqrt{x^2 + y^2}$ is odd in $x$, so its integral over the symmetric domain $D$ is zero. Thus:
\[
I = \iint_D \sin(x^2) \cos(y^2)  dx  dy
\]

Using the identity:
\[
\sin(x^2) \cos(y^2) = \frac{1}{2} [\sin(x^2 + y^2) + \sin(x^2 - y^2)]
\]

The $\sin(x^2 - y^2)$ term integrates to zero due to symmetry. Use polar coordinates:
\begin{align*}
x &= r \cos \theta, \quad y = r \sin \theta \\
I &= \frac{1}{2} \int_0^{2\pi} \int_0^{\sqrt{\pi}} \sin(r^2) \cdot r  dr  d\theta \\
&= \pi \int_0^{\sqrt{\pi}} \sin(r^2) r  dr
\end{align*}

Let $u = r^2$, $du = 2r dr$:
\[
I = \pi \int_0^{\pi} \sin u \cdot \frac{du}{2} = \frac{\pi}{2} [-\cos u]_0^{\pi} = \frac{\pi}{2} (1 - (-1)) = \pi
\]

\subsection{}
\[
I = \iint_{S} (x^{2} - x)  dy  dz + (y^{2} - y)  dz  dx + (z^{2} - z)  dx  dy
\]
where $S$ is the upper side of the upper hemisphere $x^2 + y^2 + z^2 = R^2$ ($z \geq 0$).

Use spherical coordinates:
\begin{align*}
x &= r \sin \theta \cos \phi, \quad y = r \sin \theta \sin \phi, \quad z = r \cos \theta \\
0 &\leq r \leq R, \quad 0 \leq \theta \leq \frac{\pi}{2}, \quad 0 \leq \phi \leq 2\pi
\end{align*}

\begin{align*}
I &= \int_0^{2\pi} \int_0^{\pi/2} \int_0^R (2r \cos \theta - 3) r^2 \sin \theta  dr  d\theta  d\phi \\
&= 2\pi \int_0^{\pi/2} \int_0^R (2r^3 \cos \theta - 3r^2) \sin \theta  dr  d\theta \\
&= 2\pi \int_0^{\pi/2} \left( \frac{R^4}{2} \cos \theta - R^3 \right) \sin \theta  d\theta \\
&= 2\pi \left[ -\frac{R^4}{4} \cos^2 \theta + R^3 \cos \theta \right]_0^{\pi/2} \\
&= 2\pi \left( \frac{R^4}{4} - R^3 \right) = \frac{\pi R^4}{2} - 2\pi R^3
\end{align*}

\end{document}