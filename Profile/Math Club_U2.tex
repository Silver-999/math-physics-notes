\documentclass[aspectratio=169,8pt]{beamer}
\usepackage{silence}
\WarningFilter{hyperref}{Option `driverfallback' has already been used}
\PassOptionsToPackage{unicode,CJKbookmarks}{hyperref}
\usepackage[UTF8]{ctex}
\usepackage[T1]{fontenc}
\usepackage[default,scale=0.95]{sourcesanspro}
\usefonttheme{professionalfonts}
\usetheme{Ilmenau}
\usecolortheme{whale}
\usepackage{graphicx}
\graphicspath{{figs/}}
\usepackage[absolute,overlay]{textpos}
\setbeamerfont{caption}{size=\scriptsize}
\setbeamerfont{caption name}{size=\scriptsize}
\setlength{\parskip}{0.4em}
\newcommand{\CN}[1]{{\CJKfamily{hei}#1}}
\usepackage{xcolor}
\usepackage{listings}
\usepackage{amsmath}
\usepackage{booktabs}
\lstset{
basicstyle=\ttfamily\small,
numbers=none,
breaklines=true,
showstringspaces=false,
keywordstyle=\color{blue},
commentstyle=\color{gray},
stringstyle=\color{red},
columns=fullflexible,
keepspaces=true,
texcl=false,
escapeinside={(@}{@)},
}
\lstdefinestyle{matCN}{
basicstyle=\ttfamily\scriptsize,
aboveskip=0.3ex,
belowskip=-0.7ex,
language=Matlab,
texcl=true,
morecomment=[l]\%,
morestring=[b]", 
}
\lstset{inputencoding=utf8}
\DeclareRobustCommand{\codeinfo}[1]{
\nointerlineskip
{\raggedright\color{gray}\tiny \textbf{文件:}\nolinkurl{#1}}
}
\usepackage{anyfontsize}
\renewcommand{\familydefault}{\sfdefault}
\AtBeginDocument{
  \fontsize{20}{24}\selectfont
}
\setbeamerfont{title}{size=\fontsize{28}{40}\selectfont, series=\bfseries}
\setbeamerfont{author}{size=\fontsize{22}{25}\selectfont}
\setbeamerfont{institute}{size=\fontsize{15}{22}\selectfont}
\setbeamerfont{date}{size=\fontsize{15}{17}\selectfont}

\title{期末复习第二单元}
\author{数学社}
\institute{24计量测试与仪器学院智能感知工程1班吴奕铭}
\date{2025年12月27日}

\begin{document}

\begin{frame}
  \titlepage
\end{frame}

\begin{frame}
\centering
\begin{tabular}{lcccc}
\toprule
试卷年份 & 小题分值 & 大题分值 & 总分值 \\
\midrule
2018 & 4 & 6 & 10 \\
2019 & 4 & 6 & 10 \\
2020 & 4 & 6 & 10 \\
\bottomrule
\end{tabular}
\end{frame}

\begin{frame}{1. 导数定义}
\begin{block}{定义}
\[
f^{\prime}(x_0) = \lim_{\Delta x \to 0} \frac{f(x_0 + \Delta x) - f(x_0)}{\Delta x}
\]
\end{block}
\end{frame}

\begin{frame}{2. 链式法则}
\begin{block}{公式}
\[
[f(g(x))]^{\prime} = f^{\prime}(g(x)) \cdot g^{\prime}(x)
\]
\end{block}
\end{frame}

\begin{frame}{3. 参数方程}
\[
\begin{cases}
x = x(t) \\
y = y(t)
\end{cases}
\]
\begin{block}{变换}
\[
\frac{\mathrm{d}y}{\mathrm{d}x} = \frac{\mathrm{d}y/\mathrm{d}t}{\mathrm{d}x/\mathrm{d}t}
\]
\end{block}

\end{frame}

\begin{frame}{4. 幂指函数}
\[
y = u(x)^{v(x)} \quad (u>0)
\]

\begin{block}{对数求导法}
\[
\ln y = v(x) \ln u(x)
\]
\[
y^{\prime}/y = v^{\prime} \ln u + v \cdot u^{\prime}/u
\]
\end{block}
\end{frame}

\begin{frame}{补充:三角函数的求导公式}
\huge
\[
\begin{alignedat}{2}
& (\sin x)' = \cos x, &\qquad & (\cos x)' = -\sin x, \\
& (\tan x)' = \sec^2 x, && (\cot x)' = -\csc^2 x, \\
& (\sec x)' = \sec x \tan x, && (\csc x)' = -\csc x \cot x.
\end{alignedat}
\]
\[
\begin{alignedat}{2}
& (\arcsin x)' = \frac{1}{\sqrt{1 - x^2}}, && (\arccos x)' = -\frac{1}{\sqrt{1 - x^2}}, \\
& (\arctan x)' = \frac{1}{1 + x^2}, && (\mathrm{arccot}\, x)' = -\frac{1}{1 + x^2}.
\end{alignedat}
\]
\end{frame}

\begin{frame}{补充:反函数求导法则}
\[
y = \arcsin x \ \Rightarrow\ x = \sin y
\]
\[
\frac{\mathrm{d}x}{\mathrm{d}y} = \cos y \ \Rightarrow\ \frac{\mathrm{d}y}{\mathrm{d}x} = \frac{1}{\cos y}
\]
\[
\frac{\mathrm{d}y}{\mathrm{d}x} = \frac{1}{\sqrt{1 - x^2}} \quad (-1<x<1)
\]
\end{frame}

\begin{frame}[fragile]{U2-2023-1-05}
\[
y = f\left(x^3\right),\quad \frac{\mathrm{d}^2 y}{\mathrm{d}x^2}=
\]
\end{frame}

\begin{frame}[fragile]{U2-2023-1-05}
\[
y = f\left(x^3\right),\quad \frac{\mathrm{d}^2 y}{\mathrm{d}x^2}=
\]
\begin{block}{答案}
\[
6 x f^{\prime}\left(x^3\right)+9 x^4 f^{\prime\prime}\left(x^3\right)
\]
\end{block}
\end{frame}

\begin{frame}[fragile]{U2-2023-2-02}
\[
\begin{cases}
x = \ln(1+t^2) \\
y = t - \arctan t
\end{cases}
,\quad
\frac{\mathrm{d}y}{\mathrm{d}x}=
\]
\end{frame}

\begin{frame}[fragile]{U2-2023-2-02}
\[
\begin{cases}
x = \ln(1+t^2) \\
y = t - \arctan t
\end{cases}
,\quad
\frac{\mathrm{d}y}{\mathrm{d}x}=
\]
\begin{block}{答案}
\[
= \frac{\frac{\mathrm{d}y}{\mathrm{d}t}}{\frac{\mathrm{d}x}{\mathrm{d}t}}
= \frac{(t-\arctan t)^{\prime}}{(\ln(1+t^2))^{\prime}}
= \frac{1-\frac{1}{1+t^2}}{\frac{2t}{1+t^2}}
= \frac{t}{2}
\]
\end{block}
\end{frame}

\begin{frame}[fragile]{U2-2024-1-03}
\[
\lim_{\Delta x \to 0} \frac{f(x_0-2\Delta x)-f(x_0)}{\Delta x}
\]
\end{frame}

\begin{frame}[fragile]{U2-2024-1-03}
\[
\lim_{\Delta x \to 0} \frac{f(x_0-2\Delta x)-f(x_0)}{\Delta x}
\]
\begin{block}{答案}
\[
-2f^{\prime}(x_0)
\]
\end{block}
\end{frame}

\begin{frame}[fragile]{U2-2024-2-02}
\[
e^y = 1 - xy,\quad y = y(x),\quad \left.\frac{\mathrm{d}y}{\mathrm{d}x}\right|_{x=0}
\]
\end{frame}

\begin{frame}[fragile]{U2-2024-2-02}
\[
e^y = 1 - xy,\quad y = y(x),\quad \left.\frac{\mathrm{d}y}{\mathrm{d}x}\right|_{x=0}
\]
\begin{block}{答案}
\[
y^{\prime} = - \frac{y}{x+e^y} ,\quad y^{\prime}(0)=0
\]
\end{block}
\end{frame}

\begin{frame}[fragile]{U2-2025-1-03}
\[
y = \frac{1}{2} \arctan \frac{2x}{1-x^2} ,\quad \frac{\mathrm{d}y}{\mathrm{d}x}
\]
\end{frame}

\begin{frame}[fragile]{U2-2025-1-03}
\[
y = \frac{1}{2} \arctan \frac{2x}{1-x^2}
\]
\begin{block}{答案}
\[
=\frac12\cdot\frac{1}{1+\left(\frac{2x}{1-x^2}\right)^2}\cdot\frac{2(1-x^2)-2x(-2x)}{(1-x^2)^2}
=\frac{1}{1+x^2}
\]
\end{block}
\end{frame}

\begin{frame}[fragile]{U2-2025-1-04}
\[
y = \sqrt[x]{x},\quad x>0 ,\quad y^{\prime}=
\]
\end{frame}

\begin{frame}[fragile]{U2-2025-1-04}
\[
y = \sqrt[x]{x},\quad x>0 ,\quad y^{\prime}=
\]
\begin{block}{答案}
\[
y^{\prime} = \sqrt[x]{x} \frac{1-\ln x}{x^2}
\]
\end{block}
\end{frame}

\begin{frame}[fragile]{U2}
\begin{center}
\zihao{1} 谢谢大家 \\
\vspace{1em}
\zihao{3} 第三单元,夏鑫老师有请
\end{center}
\end{frame}

\end{document}