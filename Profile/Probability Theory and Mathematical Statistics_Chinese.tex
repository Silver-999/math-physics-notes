\documentclass{article}
\usepackage[top=1.5cm, bottom=2cm, left=2cm, right=2cm]{geometry}
\usepackage{ctex}
\usepackage{amsmath, amssymb}
\usepackage{booktabs}
\usepackage{array}
\usepackage{graphicx}
\usepackage{multirow}

\begin{document}

\begin{center}
    \huge\textbf{概率论与数理统计}
\end{center}

\section*{一、概率公式}
\begin{itemize}
    \item \textbf{加法公式}:$P(A + B) = P(A \cup B) = P(A) + P(B) - P(AB)$
    \item \textbf{减法公式}:$P(A - B) = P(A\overline{B}) = P(A) - P(AB)$
    \item \textbf{条件概率}:$P(A|B) = \dfrac{P(AB)}{P(B)}$,$P(B) > 0$
    \item \textbf{相容事件}:$P(AB) > 0$
    \item \textbf{互斥事件}:$P(AB) = 0$
    \item \textbf{独立事件}:$P(AB) = P(A)P(B)$
    \item \textbf{分布函数}:
    \begin{itemize}
        \item $F(a) = P\{X \leq a\}$
        \item $P\{X < a\} = \displaystyle\lim_{x \to a^-} F(x)$
        \item $f_X(x) = \dfrac{\mathrm{d}F_X(x)}{\mathrm{d}x}$
    \end{itemize}
    \item \textbf{卷积函数}($Z=X+Y$): $f_Z(z) = \int_{-\infty}^\infty f_X(x) f_Y(z-x) \,\mathrm{d}x$
    \item \textbf{卷积函数}($Z = X - Y$):$f_Z(z) = \int_{-\infty}^{\infty} f_X(x) \, f_Y(x - z) \, \mathrm{d}x$
\end{itemize}

\section*{二、数字特征}
\begin{itemize}
    \item \textbf{数学期望}:
    \begin{itemize}
        \item 连续型:若 $X$ 的概率密度为 $f(x)$,则 $\displaystyle E(X) = \int_{-\infty}^{+\infty} x f(x) \, \mathrm{d}x$
        \item 随机变量函数:$\displaystyle E[g(X)] = \int_{-\infty}^{+\infty} g(x) f(x) \, \mathrm{d}x$
    \end{itemize}

    \item \textbf{二维情形边缘期望}:若 $(X, Y)$ 的联合概率密度为 $f(x, y)$,其边缘概率密度为:
        \begin{align*}
            f_X(x) &= \int_{-\infty}^{+\infty} f(x, y) \, \mathrm{d}y \\
            f_Y(y) &= \int_{-\infty}^{+\infty} f(x, y) \, \mathrm{d}x
        \end{align*}

    \item \textbf{方差}:
    \begin{itemize}
        \item 定义式:$D(X) = E[(X - E(X))^2]$
        \item 计算式:$D(X) = E(X^2) - [E(X)]^2$
        \item 性质:$D(CX) = C^2 D(X)$
    \end{itemize}

    \item \textbf{相关系数}:$\displaystyle \rho_{XY} = \frac{\mathrm{Cov}(X,Y)}{\sqrt{D(X)} \cdot \sqrt{D(Y)}}$

    \item \textbf{不相关的等价命题}:
    \[
    \mathrm{Cov}(X,Y) = 0 \iff \rho_{XY} = 0 \iff E(XY) = E(X)E(Y) \iff D(X+Y) = D(X) + D(Y)
    \]

    \item \textbf{协方差}:
    \begin{itemize}
        \item 定义式:$\displaystyle \mathrm{Cov}(X,Y) = E[(X - E(X))(Y - E(Y))]$
        \item 计算式:$\displaystyle \mathrm{Cov}(X,Y) = E(XY) - E(X)E(Y)$
    \end{itemize}
    
    \item \textbf{协方差性质}
    \begin{enumerate}
        \item $\mathrm{Cov}(X,Y) = \mathrm{Cov}(Y,X)$
        \item $\mathrm{Cov}(X,C) = 0$
        \item $\mathrm{Cov}(aX,bY) = ab\,\mathrm{Cov}(X,Y)$
        \item $\mathrm{Cov}(X,X) = D(X)$
        \item $\mathrm{Cov}(X+Y,Z) = \mathrm{Cov}(X,Z) + \mathrm{Cov}(Y,Z)$
        \item $D(X \pm Y) = D(X) + D(Y) \pm 2\,\mathrm{Cov}(X,Y)$
        \item $X, Y$ 相互独立 $\Rightarrow \mathrm{Cov}(X,Y) = 0$
    \end{enumerate}
\end{itemize}

\section*{三、概率分布}
\begin{itemize}
    \item \textbf{泊松分布}:$X \sim P(\lambda)$
    \begin{itemize}
        \item 结论:
        若 $X \sim P(\lambda_1)$,$Y \sim P(\lambda_2)$,且 $X$ 与 $Y$ 相互独立,则 $X + Y \sim P(\lambda_1 + \lambda_2)$
    \end{itemize}
    
    \item \textbf{正态分布}:$X \sim N(\mu, \sigma^2)$
    \begin{itemize}
        \item 标准化与概率:
        \[
        P\{a < X \leq b\} = P\left\{\frac{a - \mu}{\sigma} < \frac{X - \mu}{\sigma} \leq \frac{b - \mu}{\sigma}\right\} = \Phi\left(\frac{b - \mu}{\sigma}\right) - \Phi\left(\frac{a - \mu}{\sigma}\right)
        \]
        \item 结论:若 $Z \sim N(0,1)$,则
        \[
        \forall a > 0, \quad P\{|Z| \leq a\} = 2\Phi(a) - 1
        \]
    \end{itemize}

    \item \textbf{指数分布}:$X \sim E(\lambda)$($\lambda > 0$)
    \begin{itemize}
        \item 结论:
        \begin{itemize}
            \item $P\{X > a\} = e^{-\lambda a} \quad (a > 0)$
            \item $P\{X > s + t \mid X > s\} = P\{X > t\}$,其中 $s, t > 0$
        \end{itemize}
    \end{itemize}

\newpage

    \item \textbf{常用分布表}:
    \begin{table}[h]
        \centering
        \renewcommand{\arraystretch}{1.5}
        \begin{tabular}{|c|c|c|c|c|}
        \hline
        \textbf{分布类型} & \textbf{分布记号} & \textbf{分布律/概率密度} & \textbf{期望} & \textbf{方差} \\
        \hline
        0-1分布 & $X \sim b(1,p)$ & $P\{X = k\} = p^k (1-p)^{1-k}$ \newline $k=0,1$ & $p$ & $p(1-p)$ \\
        \hline
        二项分布 & $X \sim B(n,p)$ & $P\{X = k\} = C_n^k p^k (1-p)^{n-k}$ \newline $k=0,1,\dots,n$ & $np$ & $np(1-p)$ \\
        \hline
        泊松分布 & $X \sim \pi (\lambda)$ & $P\{X = k\} = \frac{\lambda^k}{k!} e^{-\lambda}$ \newline $k=0,1,2,\dots$ & $\lambda$ & $\lambda$ \\
        \hline
        均匀分布 & $X \sim U(a,b)$ & $f(x) = 
        \begin{cases} 
        \frac{1}{b-a}, & a < x < b \\ 
        0, & \text{其他}
        \end{cases}$ & $\frac{a+b}{2}$ & $\frac{(b-a)^2}{12}$ \\
        \hline
        正态分布 & $X \sim N(\mu,\sigma^2)$ & $f(x) = \frac{1}{\sqrt{2\pi}\sigma} e^{-\frac{(x-\mu)^2}{2\sigma^2}}$ \newline $-\infty < x < +\infty$ & $\mu$ & $\sigma^2$ \\
        \hline
        指数分布 & $X \sim E(\lambda)$ & $f(x) = 
        \begin{cases} 
        \lambda e^{-\lambda x}, & x > 0 \\ 
        0, & x \leq 0
        \end{cases}$ & $\frac{1}{\lambda}$ & $\frac{1}{\lambda^2}$ \\
        \hline
        \end{tabular}
    \end{table}
\end{itemize}

\section*{四、数理统计}
\begin{itemize}
    \item \textbf{常用统计量}:
    \begin{itemize}
        \item 样本均值:$\displaystyle \overline{X} = \frac{1}{n} \sum_{i=1}^{n} X_i$
        \item 样本方差:$\displaystyle S^2 = \frac{1}{n-1} \sum_{i=1}^{n} (X_i - \overline{X})^2$
    \end{itemize}
    
    \item \textbf{三大抽样分布}:
    \begin{itemize}
        \item $\chi^2$ \textbf{分布}:
        \begin{itemize}
            \item 定义:$X_1, \dots, X_n \stackrel{\text{i.i.d.}}{\sim} N(0,1)$,则 $\displaystyle \sum_{i=1}^n X_i^2 \sim \chi^2(n)$
            \item 性质:$E(\chi^2) = n,\; D(\chi^2) = 2n$
        \end{itemize}
        
        \item $t$ \textbf{分布}:
        \begin{itemize}
            \item 定义:$X \sim N(0,1), Y \sim \chi^2(n)$ 独立,则 $\displaystyle T = \frac{X}{\sqrt{Y/n}} \sim t(n)$
            \item 性质:概率密度 $f(t)$ 为偶函数;若 $T \sim t(n)$,则 $T^2 \sim F(1,n)$
        \end{itemize}
        
        \item $F$ \textbf{分布}:
        \begin{itemize}
            \item 定义:$X \sim \chi^2(n_1), Y \sim \chi^2(n_2)$ 独立,则 $\displaystyle F = \frac{X/n_1}{Y/n_2} \sim F(n_1, n_2)$
            \item 性质:若 $F \sim F(n_1, n_2)$,则 $\displaystyle \frac{1}{F} \sim F(n_2, n_1)$
        \end{itemize}
    \end{itemize}

    \item \textbf{统计量的性质}:
    设 $X_1, X_2, \dots, X_n$ 为来自总体 $X$ 的样本,且 $E(X) = \mu$,$D(X) = \sigma^2$,则:
        \begin{itemize}
        \item $E(X_i) = \mu,\; D(X_i) = \sigma^2,\; E(\overline{X}) = \mu,\; D(\overline{X}) = \dfrac{\sigma^2}{n}$
        \end{itemize}

\newpage
    
  \item \textbf{正态总体均值、方差的检验法}(显著性水平为$\alpha$):
    \begin{table}[h]
        \centering
        \renewcommand{\arraystretch}{1.5}
        \begin{tabular}{|c|c|c|}
            \hline
            \textbf{原假设 $H_0$} & \textbf{检验统计量} & \textbf{拒绝域} \\
            \hline
            $\mu \leq \mu_0$ & \multirow{3}{*}{$\displaystyle Z = \frac{\overline{X} - \mu_0}{\sigma / \sqrt{n}}$} & $z \geq z_\alpha$ \\
            \cline{1-1} \cline{3-3}
            $\mu = \mu_0$ & & $|z| \geq z_{\alpha/2}$ \\
            \cline{1-1} \cline{3-3}
            $\mu \geq \mu_0$ & & $z \leq -z_\alpha$ \\
            \hline
            $\mu \leq \mu_0$ & \multirow{3}{*}{$\displaystyle t = \frac{\overline{X} - \mu_0}{S / \sqrt{n}}$} & $t \geq t_\alpha(n-1)$ \\
            \cline{1-1} \cline{3-3}
            $\mu = \mu_0$ & & $|t| \geq t_{\alpha/2}(n-1)$ \\
            \cline{1-1} \cline{3-3}
            $\mu \geq \mu_0$ & & $t \leq -t_\alpha(n-1)$ \\
            \hline
            $\sigma^2 \leq \sigma_0^2$ & \multirow{3}{*}{$\displaystyle \chi^2 = \frac{(n-1)S^2}{\sigma_0^2}$} & $\chi^2 \geq \chi_\alpha^2(n-1)$ \\
            \cline{1-1} \cline{3-3}
            $\sigma^2 = \sigma_0^2$ & & $\chi^2 \leq \chi_{1-\alpha/2}^2(n-1)$ 或 $\chi^2 \geq \chi_{\alpha/2}^2(n-1)$ \\
            \cline{1-1} \cline{3-3}
            $\sigma^2 \geq \sigma_0^2$ & & $\chi^2 \leq \chi_{1-\alpha}^2(n-1)$ \\
            \hline
        \end{tabular}
    \end{table}

\end{itemize}

\section*{五、参数估计与假设检验}
\begin{itemize}
    \item \textbf{无偏性}:
    若 $E(\hat{\theta}) = \theta$,则称 $\hat{\theta}$ 是 $\theta$ 的无偏估计量。
    
    \item \textbf{有效性}:
    若 $\hat{\theta}_1$ 和 $\hat{\theta}_2$ 均为 $\theta$ 的无偏估计量,$D(\hat{\theta}_1) < D(\hat{\theta}_2)$,则称 $\hat{\theta}_1$ 比 $\hat{\theta}_2$ 更有效。
    
    \item \textbf{矩估计法}:$E(X) = \overline{X}$
    \begin{itemize}
        \item 设 $X_1,X_2,\cdots,X_n$ 是来自总体的一组样本。
        \item 总体矩等于样本矩:$\int_{-\infty}^{+\infty} x f(x) \, \mathrm{d}x = \frac{1}{n}\sum_{i=1}^n X_i$
    \end{itemize}

    \item \textbf{最大似然估计步骤}:
    \begin{enumerate}
        \item 构造似然函数 $L(\theta) = \displaystyle\prod_{i=1}^n p(x_i; \theta)$
        \item 取对数 $\ln L(\theta) = \displaystyle\sum_{i=1}^n \ln p(x_i; \theta)$
        \item 解方程 $\displaystyle\frac{\mathrm{d}[\ln L(\theta)]}{\mathrm{d}\theta} = 0$
    \end{enumerate}
    
    \item \textbf{单正态总体 $\mu$ 的置信区间}(置信水平 $1-\alpha$):
    \begin{enumerate}
        \item $\sigma^2$ 已知:$\displaystyle \left(\overline{X} - Z_{\alpha/2} \cdot \frac{\sigma}{\sqrt{n}},\; \overline{X} + Z_{\alpha/2} \cdot \frac{\sigma}{\sqrt{n}}\right)$
        \item $\sigma^2$ 未知:$\displaystyle \left(\overline{X} - t_{\alpha/2}(n-1) \cdot \frac{S}{\sqrt{n}},\; \overline{X} + t_{\alpha/2}(n-1) \cdot \frac{S}{\sqrt{n}}\right)$
    \end{enumerate}

    \item \textbf{假设检验中的两类错误}:
    \begin{itemize}
        \item \textbf{第一类错误(弃真)}:当 $H_0$ 为真时,拒绝了 $H_0$。
        \item \textbf{第二类错误(取伪)}:当 $H_0$ 为假时,接受了 $H_0$。
    \end{itemize}

\end{itemize}

\end{document}